\documentclass[10pt,a4paper,oneside]{article}
\usepackage[utf8]{inputenc}
\usepackage[english,russian]{babel}
\usepackage{amsmath}
\usepackage{amsthm}
\usepackage{amssymb}
\usepackage{enumerate}
\usepackage{stmaryrd}
\usepackage{comment}
\usepackage{cmll}
\usepackage{mathrsfs}
\usepackage{hyperref}
\usepackage[left=2cm,right=2cm,top=2cm,bottom=2cm,bindingoffset=0cm]{geometry}
\usepackage{proof}
\usepackage{tikz}
\usepackage{multicol}

\makeatletter
\newcommand{\dotminus}{\mathbin{\text{\@dotminus}}}

\newcommand{\@dotminus}{%
  \ooalign{\hidewidth\raise1ex\hbox{.}\hidewidth\cr$\m@th-$\cr}%
}
\makeatother

\usetikzlibrary{arrows,backgrounds,patterns,matrix,shapes,fit,calc,shadows,plotmarks}

\newtheorem{definition}{Определение}
\begin{document}

\begin{center}{\Large\textsc{\textbf{Теоретические домашние задания}}}\\
             \it Теория типов, ИТМО, совместно М3232-М3239 и M3332-M3339, весна 2025 года\end{center}

\section*{Домашнее задание №1: лямбда исчисление --- бестиповое и просто-типизированное}

\begin{enumerate}
\item Бесконечное количество комбинаторов неподвижной точки. Дадим следующие определения
$$\begin{array}{l}
L := \lambda abcdefghijklmnopqstuvwxyzr.r(thisisafixedpointcombinator)\\
R := LLLLLLLLLLLLLLLLLLLLLLLLLL\end{array}$$
В данном определении терм $R$ является комбинатором неподвижной точки: каков бы ни был терм
$F$, выполнено $R\ F =_\beta F\ (R\ F)$.
\begin{enumerate}
\item Докажите, что данный комбинатор --- действительно комбинатор неподвижной точки.
\item Пусть в качестве имён переменных разрешены русские буквы. Постройте аналогичное выражение
по-русски: с 32 параметрами и осмысленной русской фразой в терме $L$; покажите, что оно является
комбинатором неподвижной точки.
\end{enumerate}

\item Найдите необитаемый тип в просто-типизированном лямбда-исчислении (напомним: тип $\tau$ 
называется необитаемым, если ни для какого $P$ не выполнено $\vdash P : \tau$).

\begin{comment}
\item Напомним определение: комбинатор --- лямбда-выражение без свободных переменных. Также напомним:
$$\begin{array}{l}
S := \lambda x.\lambda y.\lambda z.x\ z\ (y\ z)\\
K := \lambda x.\lambda y.x\\
I := \lambda x.x
\end{array}$$

Известна теорема о том, что для любого комбинатора $X$ можно найти выражение $P$
(состоящее только из скобок, пробелов и комбинаторов $S$ и $K$), что $X =_\beta P$.
Будем говорить, что комбинатор $P$ \emph{выражает} комбинатор $X$ в базисе $SK$.

Косвенным аргументом (пояснением, но не доказательством!) в пользу этой теоремы являются 
два следующих соображения:
\begin{itemize}
\item теорема о замкнутости ИфИИВ: если $\vdash \varphi$, то $\vdash_\rightarrow \varphi$,
значит, если выражение имеет тип, то этот тип можно получить с помощью доказательства в
стиле Гильберта;
\item типы комбинаторов $S$ и $K$ --- это, соответственно, вторая и первая схемы аксиом.
\end{itemize}

Докажите тип следующих выражений как логическое высказывание с помощью
гильбертового вывода и, пользуясь этим доказательством как источником вдохновения, 
выразите комбинаторы в базисе $SK$:
\begin{enumerate}
\item $\lambda x.\lambda y.\lambda z.y$
\item $\lambda x.\lambda y.\lambda z.y x z$
\item $\overline{1}$
\item $Not$
\item $Xor$
\item $InR$
\end{enumerate}
\end{comment}

\item Покажите на основании следующего преобразования полноту комбинаторного базиса SKI
(проведите полное рассуждение по индукции, из которого будет следовать отсутствие 
в результате других термов, кроме SKI, бета-эквивалентность
и определённость результата для всех комбинаторов $\sigma$):

$$[\sigma]=\left\{\begin{array}{ll}
x, & \sigma = x\\
\left[\varphi\right]\ [\psi], & \sigma = \varphi\ \psi\\
K\ [\varphi], & \sigma = \lambda x.\varphi,\quad x \notin FV(\varphi)\\
I, & \sigma = \lambda x.x \\
\left[\lambda x.\left[\lambda y.\varphi\right]\right], & \sigma = \lambda x.\lambda y.\varphi,\quad x \in FV(\varphi), x \ne y\\
S\ [\lambda x.\varphi]\ [\lambda x.\psi], & \sigma = \lambda x.\varphi\ \psi,\quad x \in FV(\varphi)\cup FV(\psi)
\end{array}\right.$$

Заметим, что данные равенства объясняют смысл названий комбинаторов:

\begin{tabular}{ll}
$S$ & verSchmelzung, <<сплавление>> \\
$K$ & Konstanz \\
$I$ & Identit\"at
\end{tabular}

\item Покажите, что следующая система комбинаторов образует полный базис в бестиповом
лямбда-исчислении, но соответствующая им система аксиом в исчислении гильбертового типа
не образует полного базиса для импликативного фрагмента:

$$\begin{array}{l}
I := \lambda x.x\\
K := \lambda x.\lambda y.x\\
S' := \lambda i.\lambda x.\lambda y.\lambda z.i\ (i\ ((x\ (i\ z))\ (i\ (y\ (i\ z)))))
\end{array}$$

Указание: покажите невыводимость $(\varphi \rightarrow \varphi \rightarrow \psi) \rightarrow (\varphi \rightarrow \psi)$.

\item Напомним определение аппликативного порядка редукции:
редуцируется самый левый из самых вложенных редексов. Например, в случае выражения
$(\lambda x.I\ I)\ (\lambda y.I\ I)$ самые вложенные редексы --- применения $I\ I$:

$$(\lambda x.\underline{I\ I})\ (\lambda y.\underline{I\ I})$$

и надо выбрать самый левый из них:

$$(\lambda x.\underline{I\ I})\ (\lambda y.I\ I)$$
\begin{enumerate}
\item Проведите аппликативную редукцию выражения $2\ 2$.
\item Докажите или опровергните, что параллельная бета-редукция из теоремы Чёрча-Россера не медленнее 
(в смысле количества операций для приведения выражения к нормальной форме), чем нормальный порядок 
редукции с мемоизацией.
\item Найдите лямбда-выражение, которое редуцируется медленнее при нормальном порядке редукции,
чем при аппликативном, даже при наличии мемоизации.
\end{enumerate}

\item Сформулируем определение бета-редукции на языке пред-лямбда-термов. $A \rightarrow_\beta B$, если:
\begin{itemize}
\item $A \equiv (\lambda x.P)\ Q$, $B \equiv P\ [x := Q]$, при условии свободы для подстановки;
\item $A \equiv (P\ Q)$, $B \equiv (P'\ Q')$, при этом $P \rightarrow_\beta P'$ и $Q = Q'$, либо $P = P'$ и $Q \rightarrow_\beta Q'$;
\item $A \equiv (\lambda x.P)$, $B \equiv (\lambda x.P')$, и $P \rightarrow_\beta P'$.
\end{itemize}

\begin{enumerate}
\item Найдите лямбда-выражение, бета-редукция которого не может быть произведена из-за нарушения
правила свободы для подстановки (для продолжения редукции потребуется производить переименование
связанных переменных). Поясните, какое ожидаемое ценное свойство будет нарушено, если ограничение
правила проигнорировать.
\item Покажите, что недостаточно наложить требования на исходное выражение, и свобода для подстановки
может быть нарушена уже в процессе редукции исходно полностью корректного лямбда-выражения.
\end{enumerate}

\begin{comment}
\item Будем говорить, что выражение $A$ находится в \emph{слабой заголовочной нормальной форме} (WHNF),
если оно не имеет вид $A \equiv (\lambda x.P)\ Q$ (то есть, самый верхний терм его не является редексом).
Выражение находится в \emph{заголовочной нормальной форме} (HNF), когда его верхний терм --- не редекс и не лямбда-абстракция
с бета-редексами в теле.
\begin{enumerate}
\item Приведите нормальным порядком редукции выражение $2\ 2$ в СЗНФ.
\item Приведите нормальным порядком редукции выражение $Y\ (\lambda f.\lambda x.(IsZero\ x)\ 1\ (x \cdot f(x-1)))\ 3$ в СЗНФ.
\item Верно ли, что <<нормальность>> формы выражения может в процессе редукции только усиливаться
(никакая --- слабая заголовочная Н.Ф. --- заголовочная Н.Ф. --- нормальная форма)?
\end{enumerate}
\end{comment}

\item Две задачи на вычисление СЗНФ при помощи нормального порядка редукции.

\begin{enumerate}
\item Постройте функцию прибавления 1 к значениям из списка в лямбда-исчислении:\\ \verb!let plus1 l = map (fun x -> x+1) l!.
Постройте бесконечный список из нечётных чисел \verb![1,3,..]!. Примените функцию \verb!plus1! к списку и получите результат
в СЗНФ.

\item Напишите функцию вычисления суммы первых двух элементов списка:\\
\verb!let compute (l1::l2::ls) = (l1+l2, ls)!, примените её к результату предыдущего пункта,
получите результат в СЗНФ.
\end{enumerate}

\item Как мы уже разбирали, $\not\vdash x\ x:\tau$ в силу дополнительных ограничений
правила
$$\infer[x \notin FV(\Gamma)]{\Gamma, x:\tau\vdash x:\tau}{}$$

Найдите лямбда-выражение $N$, что $\not\vdash N:\tau$ в силу ограничений правила
$$\infer[x \notin FV(\Gamma)]{\Gamma \vdash \lambda x.N:\sigma\to\tau}{\Gamma, x:\sigma \vdash N:\tau}$$

%\item Верно ли, что $S = B(BW)(BBC)$? Если нет, то как правильно?

\item Рассмотрим подробнее отличия исчисления по Чёрчу и по Карри. 
Определим точно бета-редукцию в исчислении по Чёрчу: $A \rightarrow_\beta B$, если

\begin{tabular}{ll}
$A = (\lambda x^\sigma.P)\ Q$, & $B = P [x := Q]$\\
$A = P\ Q$, & $B = P\ Q'$ или $B = P'\ Q$ при $P \rightarrow_\beta P'$ и $Q \rightarrow_\beta Q'$\\
$A = \lambda x^\sigma.P$, & $B = \lambda x^\sigma.P'$ при $P \rightarrow_\beta P'$
\end{tabular}

\begin{enumerate}
\item Покажите, что в исчислении по Карри не выполняется свойство расширения типизации
(subject expansion) даже в такой формулировке: 
если $\vdash_\text{к} M:\sigma$, $M \twoheadrightarrow_\beta N$ и $\vdash_\text{к} N:\tau$,
то необязательно, что $\sigma=\tau$.
\item Покажите, что в исчислении по Чёрчу свойство расширения типизации в такой формулировке также не выполняется:

\begin{center}найдутся такие $M,N,\sigma$, что $\vdash_\text{ч} N:\sigma$, $M\twoheadrightarrow_\beta N$, но $\not\vdash_\text{ч} M:\sigma$.\end{center}

Но при этом в исчислении по Чёрчу выполнено свойство расширения типизации в такой формулировке:

\begin{center}если $\vdash_\text{к} M:\sigma$, $M \twoheadrightarrow_\beta N$ и $\vdash_\text{к} N:\tau$,
то тогда $\sigma=\tau$.\end{center}
\end{enumerate}
\end{enumerate}

\end{document}
