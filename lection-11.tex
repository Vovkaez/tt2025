\documentclass[aspectratio=169,dvipsnames,usenames]{beamer}
\usepackage{graphicx}
\usepackage{stmaryrd}
\usepackage[T1,T2A]{fontenc}
\usepackage[utf8]{inputenc}
\usepackage[english,russian]{babel}
\usepackage{amsmath}
\usepackage{amsfonts}
\usepackage{amssymb}
\usepackage{makeidx}
\usepackage{verbatim}
\usepackage{amsthm}
%\usepackage{bnf}
\usepackage{tikz}
\usepackage{enumerate}
\usepackage{mathtext}
\usepackage{mathtools}
\usepackage{mathabx}
%\usepackage[left=2cm,right=2cm,top=2cm,bottom=2cm,bindingoffset=0cm]{geometry}
\usepackage{proof}
%\usepackage{paracol}
%\usepackage{enumitem}
\usepackage{xcolor}
\usepackage{colortbl}
%\usepackage{minted}
%\usepackage{hyperref}
\setbeamertemplate{navigation symbols}{}
\usetikzlibrary{graphs}
\usetikzlibrary{graphs.standard}
\usetikzlibrary{automata,positioning}
\usepackage{float}

\begin{document}

%\theoremstyle{dfn}
\newtheorem{dfn}{Определение}[section]
\newtheorem{nte}{Замечание}[section]

\newtheorem{axiom}{Аксиома}[section]
\newtheorem{thm}{Теорема}[section]
\newtheorem{lmm}[theorem]{Лемма}
\newtheorem{statement}{Утверждение}[section]
\newtheorem{oun_paragraph}{Пункт}[section]
\newtheorem{cons}{Следствие}[section]
\newtheorem*{exm}{Пример}

\newcommand{\comb}[1]{\operatorname{\mathcal{#1}}}
\newcommand{\func}[1]{\operatorname{#1}}
\newcommand{\reduction}[1]{{\color{OrangeRed}#1}}
\newcommand{\set}[1]{\left\{#1\right\}}

\def\from#1{\par \parbox{0.7\textwidth}{\par \hfill\raggedleft \it #1}} 

\begin{frame}{}
\begin{center}
{\LARGE Объектно-ориентированное программирование и ТТ}
\end{center}
\end{frame}

\begin{frame}{Отношение <<быть подтипом>>}
\begin{dfn}
Будем говорить, что $A$ --- подтип $B$: $$A <: B$$
если значения типа $A$ --- часть множества значений типа $B$.
\end{dfn}
\begin{exm}
\begin{itemize}
\item В Java $\text{String} <: \text{Object}$.
\item В Java $\text{int} \not<: \text{Integer}$.
\end{itemize}
\end{exm}
\end{frame}

\begin{frame}{Ко- и контравариантность}
Определение взято из теории категорий (и упрощено):
\begin{dfn}
Пусть заданы два упорядоченных множества, $\langle C, < \rangle$ и $\langle D, < \rangle$.
Ковариантное (контравариантное) отображение --- $f: C \rightarrow D$, что 
\begin{itemize}
\item при $x<y$ всегда $f(x) < f(y)$ (ковариантное отображение);
\item при $x<y$ всегда $f(y) < f(x)$ (контравариантное отображение).
\end{itemize}
\end{dfn}
\end{frame}

\begin{frame}{Операции на типах}
Пусть $g : \star \rightarrow \star$. Что можно сказать о его ко- или контравариантности?

\begin{itemize}
\item Функции ковариантны по результату: $g(\sigma) := \text{int}\rightarrow\sigma$. 

Если $\alpha <: \beta$, $a : g(\alpha)$, $b : g(\beta)$ то $a(n): \alpha, b(n): \beta$, 
поэтому функцию $b$ можно всегда заменить на $a$. Отсюда,
$$\alpha <: \beta \text{ влечёт } g(\alpha) <: g(\beta)$$

\item Функции контравариантны по аргументу: $g(\sigma) := \sigma\rightarrow\text{int}$. 

Если $\alpha <: \beta$, $a : g(\alpha)$, $b : g(\beta)$, и $x : \alpha$ то
$a(x)$ и $b(x)$ законны. Отсюда,

$$\alpha <: \beta \text{ влечёт } g(\beta) <: g(\alpha)$$

\item Массивы инвариантны: $g(\alpha) := \alpha[]$.

С одной стороны, $g(\alpha).set : \alpha \rightarrow ()$, с другой стороны, $g(\alpha).get : \text{int} \rightarrow \alpha$.

В Java массивы ковариантны, что приводит к проверке времени исполнения.

\end{itemize}
\end{frame}

\begin{frame}[fragile]{Структурная и именная эквивалентность}
\begin{verbatim}
struct X { a: int, b: char } x;
struct Y { a: int, b: char } y;
\end{verbatim}

Структурная эквивалентность: $X \approx Y$

Именная эквивалентность: $X \not\approx Y$

\end{frame}

\begin{frame}{Исчисление $F_{<:}$}
Язык:
$$ T ::= x | \lambda x^\tau.T | T\ T | \lambda \alpha<:\tau.T| t\ \tau$$
Типы:
$$ \tau ::= \alpha|\top|\tau\rightarrow\tau|\forall\alpha<:\tau.\tau$$
Подтипизация:
$$\infer{\Gamma\vdash\sigma<:\sigma}{}\quad\quad\infer{\Gamma\vdash\sigma<:\tau}{\Gamma\vdash\sigma<:\rho\quad\quad\Gamma\vdash\rho<:\tau}\quad\quad\infer{\Gamma\vdash\sigma<:\top}{}$$
$$\infer{\Gamma,\alpha<:\tau\vdash\alpha<:\tau}{}\quad\quad\infer{\Gamma\vdash\sigma_1\rightarrow\sigma_2<:\tau_1\rightarrow\tau_2}{\Gamma\vdash\tau_1<:\sigma_1\quad\quad\Gamma\vdash\sigma_2<:\tau_2}$$


\end{frame}

\begin{frame}{Исчисление $F_{<:}$}
Типизация: правила системы $F$ со следующими добавлениями/изменениями
$$\infer[\text{тип. абстракция}]{\Gamma\vdash\lambda\alpha<:\tau_1.T_2 : \forall \alpha<:\tau_1.\tau_2}{\Gamma,\alpha <: \tau_1 \vdash T_2 : \tau_2}$$
$$\infer[\text{тип. применение}]{\Gamma\vdash T_1\ \tau_2 : \tau_{12}[\alpha := \tau_2]}{\Gamma\vdash T_1 : \forall\alpha <: \tau_{11}.\tau_{12}\quad\quad\Gamma \vdash \tau_2 <: \tau_{11}}$$
$$\infer[\text{приведение}]{\Gamma\vdash T:\tau}{\Gamma\vdash T:\sigma\quad\quad\Gamma\vdash\sigma<:\tau}$$
\end{frame}

\begin{frame}{Полное и ядерное правила}
Ядерное правило подтипизации:
$$\infer{\Gamma\vdash\forall\alpha<:\rho_1.\sigma_2<:\forall\alpha<:\rho_1.\tau_2}{\Gamma,\alpha<:\rho_1\vdash\sigma_2<:\tau_2}$$
Полное правило подтипизации:
$$\infer{\Gamma\vdash\forall\alpha<:\sigma_1.\sigma_2<:\forall\alpha<:\tau_1.\tau_2}{\Gamma\vdash\tau_1<:\sigma_1\quad\quad\Gamma,\alpha<:\tau_1\vdash\sigma_2<:\tau_2}$$
\end{frame}

\begin{frame}{Типизация для пар}
Можно показать:
$$\infer{\sigma_1\&\sigma_2 <: \tau_1\&\tau_2}{\Gamma\vdash\sigma_1<:\tau_1\quad\quad\Gamma\vdash\sigma_2<:\tau_2}$$
\end{frame}

\begin{frame}{Что такое класс/объект}
\begin{itemize}
\item Определим отношение подтипизации на кортежах --- это даст наследование.
$$\text{struct }x_1 : \tau_1,\dots,x_n : \tau_n\text{ end} ::= \langle x_1 : \tau_1, \langle\dots\langle x_n : \tau_n, T : \top\rangle\rangle$$
Упростим, убрав имена полей (не теряя общности):
$$\text{struct }\tau_1,\dots,\tau_n\text{ end} ::= \tau_1\&\dots(\tau_n\&\top)$$
%Можно показать, что:
%$$\infer{\sigma_1\&\sigma_2 <: \tau_1\&\tau_2}{\Gamma\vdash\sigma_1<:\tau_1\quad\quad\Gamma\vdash\sigma_2<:\tau_2}$$
Тогда $$\tau_n\&\tau_{n+1} <: \tau_n\&\top$$
Отсюда, $$\text{struct }x_1 : \tau_1,\dots,x_n : \tau_n\text{ end} <: \text{struct }x_1 : \tau_1,\dots,x_{n-1} : \tau_{n-1}\text{ end}$$
\item Приведения типов и т.п. получаются согласно общим правилам $F_{<:}$.
\item По необходимости добавим экзистенциальные типы.
\item При необходимости сделать именную эквивалентность из структурной можно, включив какие-нибудь токены в тип .
\end{itemize}
\end{frame}

\begin{frame}{Objective Caml}
\begin{itemize}
\item ML --- Meta Language, Робин Милнер, 1970е.
\item Standard ML, 1983.
\item Category Abstract Machine Language (Caml), 1985.
\item ZINC project (ZINC Is Not CAML): ``Toplevels considered harmful'', Ксавье Леруа (Xavier Leroy), 1990.
\item Objective Caml, 1996.
\end{itemize}
\end{frame}

\begin{frame}{Объекты, классы, подтипы}
\begin{itemize}
\item Объектно-ориентированность --- набор различных конструкций, которые можно собирать по-разному.
\item Объектно-ориентированность в Окамле собрана иначе, чем в Джаве или в Смолтоке.
\item В Окамле есть две различных конструкции: модули (соответствуют абстрактным типам данных), 
и объекты и классы (предназначены для построения иерархии наследования).
\item Отношение подтипирования определено независимо от модулей и классов, в том числе и на обычных типах.
\end{itemize}
\end{frame}

\begin{frame}[fragile]{Полиморфный вариантный тип}
Традиционный алгебраический тип не допускает пересечение вариантов:
\begin{verbatim}
type abc = A | B | C;;
type ac = A | C;;             (* ac.C <> abc.C *)
\end{verbatim}

Однако, есть полиморфный вариантный тип:
\begin{verbatim}
type abc = [`A | `B | `C];;
type ac = [`A | `C ];;
\end{verbatim}

Заметим, что $[>`A|`C] :> [>`A|`B|`C]$.

\vspace{0.5cm}
Автоматический вывод типов не может построить правильное подтипирование, но можно указать его явно:

\begin{verbatim}
let f a = `A;;                  (* f : 'a -> [> `A ] *)
let g = (f () : ac);;           (* ошибка: у типа f() нет тэга `C *)
let g = (f () : [`A] :> ac);;   (* g : ac, явное приведение типа *)
\end{verbatim}
\end{frame}

\begin{frame}[fragile]{Модули: ковариантный интерфейс}
\footnotesize
\begin{verbatim}
module type Writer = sig
    type +'b t
    val empty : unit -> 'b t
    val push : 'b t -> 'b -> 'b t
    val print_len : 'b t -> unit
end

module Writer : Writer = struct
    type 'b t = 'b list
    let empty () = []
    let push l x = x :: l
    let print_len l = print_int (List.length l)
end

let w = Writer.empty ();;
let w = Writer.push w `A;;
let w = Writer.push w `C;;
let u = (w : [`A|`C] Writer.t :> [`A|`C|`E] Writer.t);;
let u = Writer.push u `E;;
\end{verbatim}
%Writer.print_len u;;
\end{frame}

\begin{frame}[fragile]{Модули: контравариантный интерфейс}
\footnotesize
\begin{verbatim}
module type Counter = sig
    type -'b t
    val empty : ('b->int->int) -> 'b t
    val push : 'b t -> 'b -> 'b t
    val print_len : 'b t -> unit
end

module Counter : Counter = struct
    type 'b t = int * ('b->int->int)
    let empty f = (0,f)
    let push (l,f) x = (f x l, f)
    let print_len (l,f) = print_int l
end

let w = Counter.empty (fun x l -> l + match x with `A -> 0 | `B -> 1 | `C -> 2);;
let w = Counter.push w `A;;
let w = Counter.push w `C;;
let u = (w : [`A|`C] Counter.t :> [`A] Counter.t);; (* однако, [`A] :> [`A|`C] *)
let u = Counter.push u `A;;
\end{verbatim}
%Counter.print_len u;;
\end{frame}

\begin{frame}[fragile]{Объекты}
Зададим объект:

\footnotesize
\begin{verbatim}
type square = < area : float; width : int >;;

let square w = object
  method area = Float.of_int (w * w)
  method width = w
end;;
\end{verbatim}

\normalsize
И нечто общее:

\footnotesize
\begin{verbatim}
let coin = object
  method shape = circle 5
  method color = "silver"
end;;

let map = object
  method shape = square 10
end;;

type item = < shape : shape >;;
let items = [ (coin :> item) ; (map :> item) ];;
\end{verbatim}
\end{frame}

\begin{frame}{Формализация: подтипы}
\begin{itemize}
\item ``Быть подтипом'' определяется рекурсивно --- согласно структуре.
\item Приведение типа:
$$\infer{\Gamma\vdash(\alpha: \tau<:\tau'):\theta(\tau')}{\tau <: \tau'\quad\quad\Gamma\vdash\alpha:\theta(\tau)}$$
\end{itemize}
\end{frame}

\begin{frame}[fragile]{Формализация: экзистенциальные типы и модули}
\begin{itemize}
\item 
\begin{verbatim}
module type Counter = sig
    type -'b t
    val empty : ('b->int->int) -> 'b t
    ...
end
\end{verbatim}
То есть, $\exists \beta.E : (\beta\rightarrow N \rightarrow N) \rightarrow \tau(\beta)$
\item Заменим экзистенциальный тип (интерфейс модуля) на его каноническое представление:
$(B\rightarrow N \rightarrow N)\rightarrow \tau(B)$
\item Правила подтипирования для типа, на основании его канонического представления:
$$ \infer{\Gamma\vdash\exists\alpha.R <: \exists\alpha'.R}{\Gamma,\alpha\vdash R<: R'[\alpha':=\tau]}$$
\end{itemize}
\end{frame}

\end{document}
