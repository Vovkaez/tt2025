\documentclass[aspectratio=169,dvipsnames,usenames]{beamer}
\usepackage{graphicx}
\usepackage{stmaryrd}
\usepackage[T1,T2A]{fontenc}
\usepackage[utf8]{inputenc}
\usepackage[english,russian]{babel}
\usepackage{amsmath}
\usepackage{amsfonts}
\usepackage{amssymb}
\usepackage{makeidx}
\usepackage{verbatim}
\usepackage{amsthm}
%\usepackage{bnf}
\usepackage{tikz}
\usepackage{enumerate}
\usepackage{mathtext}
\usepackage{mathtools}
\usepackage{mathabx}
%\usepackage[left=2cm,right=2cm,top=2cm,bottom=2cm,bindingoffset=0cm]{geometry}
\usepackage{proof}
%\usepackage{paracol}
%\usepackage{enumitem}
\usepackage{xcolor}
\usepackage{colortbl}
\usepackage{cmll}
%\usepackage{minted}
%\usepackage{hyperref}
\setbeamertemplate{navigation symbols}{}
\usetikzlibrary{graphs}
\usetikzlibrary{graphs.standard}
\usetikzlibrary{automata,positioning}
\usepackage{float}

\begin{document}

%\theoremstyle{dfn}
\newtheorem{dfn}{Определение}[section]
\newtheorem{nte}{Замечание}[section]

\newtheorem{axiom}{Аксиома}[section]
\newtheorem{thm}{Теорема}[section]
\newtheorem{lmm}[theorem]{Лемма}
\newtheorem{statement}{Утверждение}[section]
\newtheorem{oun_paragraph}{Пункт}[section]
\newtheorem{cons}{Следствие}[section]
\newtheorem*{exm}{Пример}

\newcommand{\comb}[1]{\operatorname{\mathcal{#1}}}
\newcommand{\func}[1]{\operatorname{#1}}
\newcommand{\reduction}[1]{{\color{OrangeRed}#1}}
\newcommand{\set}[1]{\left\{#1\right\}}

\def\from#1{\par \parbox{0.7\textwidth}{\par \hfill\raggedleft \it #1}} 

\begin{frame}{}
\begin{center}
{\LARGE Множества, Сетоиды, Теорема Диаконеску}
\end{center}
\end{frame}

\begin{frame}[fragile]{Set --- не множество}
\begin{axiom}[выбора]Пусть задано семейство непустых множеств $A$
($\forall a \in A.\exists y \in a$).
Тогда найдётся функция, возвращающая по элементу из каждого множества
($\exists f: A \rightarrow \cup A. f(a) \in a$).
\end{axiom}

Закодируем семейство с помощью отношения $R$: $R(a,b)$, если $b \in a$.

\begin{verbatim}
Theorem choice :
 forall (A B : Type) (R : A->B->Prop),
   (forall x : A, exists y : B, R x y) ->
    exists f : A->B, (forall x : A, R x (f x)).
\end{verbatim}

К сожалению, \verb!choice! доказуем ($f$ определяется как значения, которые возвращает конструктивный квантор существования).
\end{frame}

\begin{frame}[fragile]{Сетоид}
\begin{dfn}Сетоид --- сет с отношением эквивалентности\end{dfn}

Такое определение делает равенство невычислимым.

Почему аксиома выбора перестаёт быть доказуемой?
Пусть $x \in A, y \in A, R(x,y)$. Но тогда должно быть $f(x) = f(y)$, хотя
квантор может подсказывать разные значения
($\exists p.R(x,p)$ и $\exists q.R(y,q)$ и $p \ne q$).
\end{frame}

\begin{frame}[fragile]{}
\small
\begin{verbatim}
Reflx: {A : Type} -> (R: A -> A -> Type) -> Type
Reflx {A} R = (x : A) -> R x x

Symm: {A : Type} -> (R: A -> A -> Type) -> Type
Symm {A} R = (x : A) -> (y : A) -> R x y -> R y x

Trans: {A : Type} -> (R: A -> A -> Type) -> Type
Trans {A} R = (x : A) -> (y : A) -> (z : A) -> R x y -> R y z -> R x z

data IsEquivalence: {A : Type} -> (R: A -> A -> Type) -> Type where
    EqProof: {A: Type} -> (R: A -> A -> Type) -> 
       Reflx {A} R -> Symm {A} R -> Trans {A} R -> IsEquivalence {A} R

record Setoid where
    constructor MkSetoid
    Carrier: Type
    Equiv: Carrier -> Carrier -> Type
    EquivProof: IsEquivalence Equiv
\end{verbatim}
\end{frame}

\begin{frame}[fragile]{}
\footnotesize
\begin{verbatim}
data Map: (A:Setoid) -> (B:Setoid) -> Type where
  MkMap: {A:Setoid} -> {B:Setoid} -> (f: (Carrier A) -> (Carrier B)) -> 
     ({x:Carrier A} -> {y:Carrier A} -> 
     ((Equiv A) x y) -> ((Equiv B) (f x) (f y))) -> Map A B

MapF: {A:Setoid} -> {B:Setoid} -> Map A B -> (Carrier A -> Carrier B)
MapF (MkMap {A} {B} f ext) = f

MapExt: {A:Setoid} -> {B:Setoid} -> (p: Map A B) -> 
     ({x:Carrier A} -> {y:Carrier A} -> ((Equiv A) x y) -> ((Equiv B) (MapF p x) (MapF p y)))
MapExt (MkMap {A} {B} f ext) = ext

Rel: Type -> Type -> Type
Rel a b = a -> b -> Type

postulate ext_ac: {I: Setoid} -> {S: Setoid} -> 
  (A: Rel (Carrier I) (Carrier S)) -> 
  ((x: Carrier I) -> (g : Carrier S ** A x g)) ->
  (chs: (Map I S) ** ((w: Carrier I) -> A w ((MapF chs) w)))

excluded_middle: (P: Type) -> Or P (Not P)
\end{verbatim}
\end{frame}

\begin{frame}{Теорема Диаконеску}
\begin{thm}Если рассмотреть ИИП с ZFC, то для любого $P$ выполнено $\vdash P \vee \neg P$.\end{thm}
\begin{proof}Рассмотрим $\mathcal{B} = \{0,1\}$, $A_0 = \{ x \in \mathcal{B} | x = 0 \vee P \}$ и 
$A_1 = \{ x \in \mathcal{B} | x = 1 \vee P\}$.
$\{A_0,A_1\}$ --- семейство непустых множеств, и по акс. выбора существует
$f: \{A_0,A_1\} \rightarrow \cup A_i$, что $f(A_i) \in A_i$. (Если $P$, то $A_0 = A_1$ и $\{A_0,A_1\} = \{\mathcal{B}\}$).

\vspace{0.3cm}
\begin{tabular}{ll}
$\vdash f(A_0) \in A_0 \with f(A_1) \in A_1$ & а.выбора: $f(A_i) \in A_i$\\
$\vdash {\color{olive}f(A_0) \in \mathcal{B}} \with (f(A_0) = 0 \vee P) \with {\color{olive}f(A_1) \in \mathcal{B}} \with (f(A_1) = 1 \vee P)$ & а.выделения\\
%$\vdash(f(A_0) = 0 \vee P) \with (f(A_1) = 1 \vee P)$ & Удал. $(\with)$\\
$\vdash (f(A_0) = 0 \with f(A_1) = 1) \vee P$ & Удал. $(\with)$ + дистр.\\
$\vdash P\vee{\color{blue}f(A_0) \ne f(A_1)}$ & $0 \ne 1$ и транз.\\\pause
$\vdash P \rightarrow A_0 = A_1$ & Определение $A_i$\\
$\vdash A_0 = A_1 \rightarrow f(A_0) = f(A_1)$ & Конгруэнтность\\
$\vdash \color{blue} f(A_0) \ne f(A_1) \rightarrow \neg P$ & Контрапозиция\\
$\vdash P \vee \neg P$ & Подставили
\end{tabular}
\end{proof}
\end{frame}

\begin{frame}[fragile]{Аксиома выбора в HoTT}
\begin{dfn}Обозначим пропозициональное усечение типа $T$ как $\| T\|$\end{dfn}
%\begin{axiom}[выбора в HoTT]
Рассмотрим $B: E \rightarrow \texttt{Set}$ --- некоторое семейство множеств.
Заметим, что выражение $\Pi x^E.\| B x \|$ означает непустоту каждого элемента семейства.
Тогда следующее выражение задаёт аксиому выбора:

$$\Pi x^E.\| B x \| \rightarrow \| \Pi x^E.B x \|$$

\begin{verbatim}
\class Choice \extends BaseSet {
  | choice {B : E -> \Set} : 
       (\Pi (x : E) -> TruncP (B x)) -> TruncP (\Pi (x : E) -> B x)
}
\end{verbatim}
%\end{axiom}
\end{frame}

\begin{frame}[fragile]{Теорема Диаконеску в Аренде}
Вспомогательные определения:

\begin{verbatim}
\truncated \data Quotient {A : \Type} (R : A -> A -> \Type) : \Set
\end{verbatim}

Вспомним $A$ из теоремы Диаконеску:\\
$A_0 := \{ x \in \mathcal{B} | x = 0 \vee P \}$, $A_1 := \{ x \in \mathcal{B} | x = 1 \vee P\}$.
$A := \{A_0,A_1\}$
%// --- семейство непустых множеств, и по акс. выбора существует

$$R(x,y) := \left\{\begin{array}{ll}\text{И},&x = y\\
                                    P,&x \ne y\end{array}\right.
$$

Иными словами, $A := \mathcal{B}/_R$, то есть \verb!Quotient {Bool} R!
\end{frame}

%\begin{frame}[fragile]{Аксиома выбора в Аренде}
%Аксиома 
%\begin{verbatim}
%\axiom alem (P : \Prop) : Dec P
%\end{verbatim}
%\end{frame}

\end{document}
