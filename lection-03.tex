\documentclass[aspectratio=169]{beamer}
\setbeamertemplate{navigation symbols}{}
\usepackage{graphicx}
\usepackage{stmaryrd}
\usepackage[T1,T2A]{fontenc}
\usepackage[utf8]{inputenc}
\usepackage[english,russian]{babel}
\usepackage{amsmath}
\usepackage{amsfonts}
\usepackage{amssymb}
\usepackage{makeidx}
\usepackage{verbatim}
\usepackage{amsthm}
\usepackage{bnf}
\usepackage{tikz}
\usepackage{enumerate}
\usepackage{mathtext}
\usepackage{mathtools}
\usepackage{mathabx}
%\usepackage[left=2cm,right=2cm,top=2cm,bottom=2cm,bindingoffset=0cm]{geometry}
\usepackage{proof}
%\usepackage{paracol}
%\usepackage{enumitem}
\usepackage{color}
\usepackage{colortbl}
\usepackage{minted}
%\usepackage{hyperref}
\usetikzlibrary{graphs}
\usetikzlibrary{graphs.standard}
\usetikzlibrary{automata,positioning}
\usepackage{float}

\begin{document}

%\theoremstyle{dfn}
\newtheorem{dfn}{Определение}[section]
\newtheorem{nte}{Замечание}[section]

\newtheorem{axiom}{Аксиома}[section]
\newtheorem{thm}{Теорема}[section]
\newtheorem{lmm}[theorem]{Лемма}
\newtheorem{statement}{Утверждение}[section]
\newtheorem{oun_paragraph}{Пункт}[section]
\newtheorem{cons}{Следствие}[section]
\newtheorem*{exm}{Пример}

\newcommand{\comb}[1]{\operatorname{\bf{\textrm{#1}}}}
\newcommand{\func}[1]{\operatorname{#1}}
\newcommand{\reduction}[1]{{\color{OrangeRed}#1}}
\newcommand{\set}[1]{\left\{#1\right\}}

\def\from#1{\par \parbox{0.7\textwidth}{\par \hfill\raggedleft \it #1}} 

\begin{frame}{}
\begin{center}\Large Лекция 3.\\ \Large Система F\end{center}
\end{frame}

\begin{frame}{ИИП второго порядка}
\begin{itemize}
\item Алфавит: $a-z$, $\vee$, $\&$, $\rightarrow$, $\neg$, $\forall$, $\exists$.
\item Метапеременные: $\alpha$ для формул, $p,x,y,z$ для переменных.
\item $F ::= p\ |\ (F\star F)\ |\ (\forall p.F)\ |\ (\exists p.F)\ |\ \bot$
\item Сокращения записи: приоритеты как в ИИВ, подкванторное выражение продолжается направо настолько, насколько возможно.
\end{itemize}

\begin{exm}
$\forall p.\forall q.p \rightarrow q \rightarrow p$
\end{exm}
\end{frame}

\begin{frame}{Теория доказательств}
    Правила вывода совпадают с правилами для ИИВ, добавлены 4 новых:

    \[ \dfrac{\Gamma\vdash\varphi}{\Gamma\vdash\forall p.\varphi} (p\notin FV(\Gamma)) \qquad
        \dfrac{\Gamma\vdash\forall p.\varphi}{\Gamma\vdash\varphi[p:=\theta]} \]


    \[ \dfrac{\Gamma\vdash\varphi[p := \theta]}{\Gamma\vdash\exists p.\varphi} \qquad
       \dfrac{\Gamma\vdash\exists p.\varphi\quad\Gamma,\varphi\vdash\psi}{\Gamma\vdash\psi} (p \notin FV(\Gamma,\psi)) \]
\end{frame}

\begin{frame}{Теория моделей}
Простая неполная модель.

$$V = \{ \text{И}, \text{Л} \}$$

%\begin{center}
%	 	$\llbracket p\rrbracket=p$, т. е. $\llbracket p\rrbracket^{p = x} = x$ \\
%\end{center}
 	
 	
 \begin{center}
 		\begin{equation*}
 		\llbracket \varphi\rightarrow\psi\rrbracket = 
 		\begin{cases}
 			\text{Л}, \llbracket\varphi\rrbracket = \text{И}, \llbracket\psi\rrbracket = \text{Л} \\
 			\text{И}, \text{иначе}
 		\end{cases}
 	\end{equation*}
 \end{center}
 	
 	
 	\begin{equation*}
 		\llbracket\forall p.\varphi\rrbracket = 
 		\begin{cases}
 			\text{И}, \llbracket \varphi\rrbracket^{p:=\text{Л, И}} = \text{И} \\
 			\text{Л}, \text{иначе}
 		\end{cases}
 	\end{equation*}
% 	Эта модель корректна, но не полна.


\end{frame}

\begin{frame}[fragile]{Выразимость всех связок через $\forall$, $\rightarrow$}
Заметим, что достаточно определить связки $\forall$ и $\rightarrow$.\vspace{0.5cm}

\begin{tabular}{ll}
 Связка & Способ выразить \\
 $\alpha\&\beta$ & $\forall p.(\alpha\rightarrow\beta\rightarrow p)\rightarrow p$\\
 $\alpha\vee\beta$ & $\forall p.(\alpha\rightarrow p)\rightarrow (\beta\rightarrow p) \rightarrow p$\\
 $\bot$ & $\forall p.p$ \\
 $\exists p.\varphi$ & $\forall f.(\forall p.\varphi\rightarrow f) \rightarrow f$
\end{tabular}

\vspace{0.5cm}
С так определёнными связками оказывается возможно показать все правила вывода.
Например, примем $\alpha\&\beta$ за $\forall p.(\alpha\rightarrow\beta\rightarrow p)\rightarrow p$ и покажем,
что из $\alpha\&\beta$ следует $\alpha$:
$$\infer{\alpha}{
    \infer{\vdash\alpha\rightarrow\beta\rightarrow\alpha}{\infer{\alpha\vdash\beta\rightarrow\alpha}{\infer{\alpha,\beta\vdash\alpha}{}}}\quad\quad
    \infer[p := \alpha]{\vdash(\alpha\rightarrow\beta\rightarrow \alpha)\rightarrow \alpha}{\vdash\forall p.(\alpha\rightarrow\beta\rightarrow p)\rightarrow p}
}$$

\end{frame}

\begin{frame}[fragile]{Система F}
\begin{dfn}
        Типы в системе F:
 	\begin{equation*}
 	\tau =
 	\begin{cases}
 	\alpha,\beta,\gamma ...\quad(\text{атомарные типы}) \\
 	\tau\rightarrow\tau \\
 	\forall\alpha.\tau\qquad(\alpha\text{ - переменная})
 	\end{cases}
 	\end{equation*}
 \end{dfn}

 \begin{dfn}
        Пред-лямбда-терм в системе F (типизировано по Чёрчу):
 	\begin{center}
 		$F ::= x\ |\ (\lambda x^{\tau}.F)\ |\ (F\ F)\ |\ (\Lambda\alpha.F)\ |\ (F\ \tau)$ 
 	\end{center}
 \end{dfn}
\end{frame}

\begin{frame}[fragile]{Типовая абстракция и применение}
   Примеры соответствующих конструкций из C++.
   \begin{itemize}
    \item Типовая абстракция, $\Lambda \tau.W$:
    \begin{verbatim}
template<typename t>
class W {
    t x;
}
    \end{verbatim}
 	
    \item Типовое применение, $W\ int$:
    \begin{verbatim}
W<int> w_test;
    \end{verbatim}
  \end{itemize}
\end{frame}
 	
\begin{frame}
 	В системе F определены следующие правила вывода: \\ \\
  	\noindent
 	{$\dfrac{}{\Gamma,x:\tau\vdash x:\tau}\qquad\qquad$} 
 	{$\dfrac{\Gamma\vdash M:\sigma\rightarrow\tau\qquad\Gamma\vdash N:\sigma}{\Gamma\vdash M N:\tau}$}\\  \\ \\
 	{$\dfrac{\Gamma,x:\tau\vdash M:\sigma}{\Gamma\vdash\lambda x^{\tau}.M:\tau\rightarrow\sigma}\quad(x\notin FV(\Gamma))$}\\ \\ \\
 	{$\dfrac{\Gamma\vdash M:\sigma}{\Gamma\vdash\Lambda\alpha.M:\forall\alpha.\sigma}\quad(\alpha\notin FV(\Gamma))\qquad$}
 	 $\dfrac{\Gamma\vdash M:\forall\alpha.\sigma}{\Gamma\vdash M\tau:\sigma[\alpha:=\tau]}$
 	\\
 	
    \emph{Начнем с $\beta$-редукции:}
    \begin{enumerate}
        \item Типовая $\beta$-редукция: $(\Lambda\alpha.M^{\sigma})\tau\rightarrow_\beta M[\alpha:= \tau]:\sigma[\alpha:= \tau]$
        \item Классическая $\beta$-редукция: $(\lambda x^{\sigma}.M)^{\sigma\rightarrow\tau}X\rightarrow_\beta M[x:=X]:\tau$ 
    \end{enumerate}
 	
\end{frame}

\begin{frame}[fragile]{Абстрактные типы данных}
	%Объектно-ориентированные типы данных = абстрактные типы данных + наследование.
	%Значит, АТД = ООП - наследование.
	%Напомним, запись $a.method(...)$ --- другая запись для $method (a, ...)$.

	Стек $\alpha$ из значений типа $\upsilon$: контейнер, соответствующий интерфейсу

	\begin{center}\begin{tabular}{lll}
	метод & тип & комментарий\\\hline
	\verb!empty! & $\alpha$ & (конструктор)\\
	\verb!push! & $\upsilon \rightarrow \alpha \rightarrow \alpha$\\
	\verb!pop! & $\alpha \rightarrow \alpha \& \upsilon$
	\end{tabular}\end{center}

	%Стек: $s = empty(); s.push(5); s.push(12); while (!s.empty()) s.pop()$
	Возможны разные реализации интерфейса.\vspace{0.5cm}

	\textbf{Замечание}:  Мы понимаем АТД как набор функций, без собственных данных.
	Напомним, что \verb!a.method(...)! --- другая запись для \verb!method(a, ...)!.
\end{frame}

\begin{frame}[fragile]{Пример определения и применения АТД}
\begin{verbatim}
abstype stack with 
    empty : stack
    push : int * stack -> stack
    pop : stack -> stack * int
is pack Maybe Int,
    empty = None
    push (n,s) = Some n
    pop s = case s with None -> 0 | Some v -> v
in
    stack::pop(stack::push(12,stack::empty))
\end{verbatim}
\end{frame}

\begin{frame}{Экзистенциальные типы}
	Экзистенциальный тип --- тип, соответствующий квантору существования в смысле изоморфизма Карри-Ховарда.
	Соответствует абстрактному типу данных.

	 $$\dfrac{\Gamma\vdash\varphi[\alpha:=\theta]}{\Gamma\vdash\exists\alpha.\varphi}\quad\quad
	  \dfrac{\Gamma\vdash\exists\alpha.\varphi\qquad\Gamma,\varphi\vdash\psi}{\Gamma\vdash\psi}$$

	АТД имеет интерфейс $\varphi$, тип АТД $\alpha$ реализуется типом $\theta$, а сам интерфейс --- термом $M$:

	$$\dfrac{\Gamma \vdash M : \varphi[\alpha := \theta]}{\Gamma\vdash (\text{pack } M, \theta \text{ to } \exists \alpha . \varphi) : \exists \alpha.\varphi}$$

	... и если вычисление $N:\psi$ работает при условии наличия какой-то реализации АТД $x:\varphi$ в контексте, то нам достаточно
	АТД $P:\exists\alpha.\varphi$ для получения результата:

	$$\dfrac{\Gamma \vdash P : \exists \alpha . \varphi\qquad\Gamma, x : \varphi \vdash N : \psi}{\Gamma \vdash \text{abstype } \alpha \text{ with } x:\varphi \text{ is } P \text{ in } N:\psi}
	(\alpha \notin FV(\Gamma, \psi))$$


	%  Предположим, что у нас есть стек с хранилищем типа $\alpha$, у которого определены следующие операции:\\
	% \textbf{empty}: $\alpha$\\
	% \textbf{push}: $\alpha\&\nu\rightarrow\alpha$\\
	% \textbf{pop}: $\alpha\rightarrow\alpha\&\nu$\\
	 
	% Интерфейс стека: stack$\equiv\alpha\&(\alpha\&\nu\rightarrow\alpha)\&(\alpha\rightarrow\alpha\&\nu)$.
        %Что такое $\alpha$? Неважно, детали опустим:\\$\exists\alpha.\alpha\&(\alpha\&\nu\rightarrow\alpha)\&(\alpha\rightarrow\alpha\&\nu)$
\end{frame}
	 
\begin{frame}[fragile]{Стек в $F$}

	$$\dfrac{\Gamma \vdash M : \varphi[\alpha := \theta]}{\Gamma\vdash (\text{pack } M, \theta \text{ to } \exists \alpha . \varphi) : \exists \alpha.\varphi}\quad
	\dfrac{\Gamma \vdash P : \exists \alpha . \varphi\qquad\Gamma, x : \varphi \vdash N : \psi}{\Gamma \vdash \text{abstype } \alpha \text{ with } x:\varphi \text{ is } P \text{ in } N:\psi}$$
%	(\alpha \notin FV(\Gamma, \psi))$$

	Интерфейс стека (возьмём $\upsilon$ как чёрчевский нумерал):

	$$\varphi := (\underbrace{\alpha}_{empty}\&\underbrace{(\upsilon\&\alpha\rightarrow\alpha)}_{push})\&\underbrace{(\alpha\rightarrow\alpha\&\upsilon)}_{pop}$$

	Какое-нибудь вычисление --- скажем, \verb!pop(push(12,empty))!:

	$$x : \varphi\vdash \underbrace{\pi_R \left((\pi_R x) ((\pi_R (\pi_L x)) \langle 12, \pi_L (\pi_L x)\rangle)\right)}_{N}:\upsilon$$

	И простая реализация, для $\theta := (\gamma\rightarrow\gamma)\vee\upsilon$ --- это \verb!Maybe Int!:

	$$\vdash \langle \langle (In_L\ \lambda x.x), \lambda n.In_R\ (\pi_L\ n)\rangle, \lambda n.\texttt{case }(\lambda x.0)\ (\lambda x.x)\ n\rangle : \varphi[\alpha := \theta]$$
%$\vdash M : \varphi$

	$ $

        %let (stk, v) = pop (push 12 $ push 5 empty) in
        %let (stk2, v2) = pop stk in 
        %v + v2

\end{frame}

\begin{frame}{Раскрываем $\exists$ через $\forall$}
Напомним, что $\exists\alpha.\varphi := \forall \beta.(\forall\alpha.\varphi\rightarrow\beta)\rightarrow \beta$.

	$$\dfrac{\Gamma \vdash M : \varphi[\alpha := \theta]}{\Gamma\vdash (\text{pack } M, \theta \text{ to } \exists \alpha . \varphi) : \exists \alpha.\varphi}$$

Перепишем это правило только через базовые конструкции системы $F$:

$$\dfrac{\Gamma\vdash M : \varphi[\alpha:=\theta]}{\Gamma\vdash\Lambda \beta.\lambda e^{\forall \alpha.\varphi\rightarrow\beta}.(e\ \theta)\ M : \forall \beta.(\forall \alpha.\varphi\rightarrow\beta)\rightarrow \beta}$$

<<Пусть есть вычисление $e$, использующее АТД $\alpha$ с интерфейсом $\varphi$, возвращающее $\beta$. 
Тогда, имея конкретный тип реализации АТД $\theta$ и саму реализацию АТД $M : \varphi[\alpha:=\theta]$,
то с помощью вычисления $e$ возможно вычислить результат и вернуть значение типа $\beta$>>. 

Сравните с \texttt{case} для алгебраического типа и вспомните действия редактора связей (линкера).
%(\Lambda \alpha.\lambda m^\varphi.M\ m)\ M$

\end{frame}

\begin{frame}{Раскроем \texttt{abstype}}

	$$\dfrac{\Gamma \vdash P : \exists \alpha . \varphi\qquad\Gamma, x : \varphi \vdash N : \psi}{\Gamma \vdash \text{abstype } \alpha \text{ with } x:\varphi \text{ is } P \text{ in } N:\psi}$$

Перепишем это правило через базовые конструкции системы $F$:

$$\dfrac{\Gamma\vdash P:\forall \beta.(\forall \alpha.\varphi\rightarrow\beta)\rightarrow\beta\qquad\Gamma,x:\varphi\vdash N:\psi}
        {\Gamma\vdash (P\ \psi)\ (\Lambda \alpha.\lambda x^\varphi.N) : \psi}$$

Вспомним pack:

$$\dfrac{\Gamma\vdash M : \varphi[\alpha:=\theta]}{\Gamma\vdash\Lambda \beta.\lambda e^{\forall \alpha.\varphi\rightarrow\beta}.(e\ \theta)\ M : \forall \beta.(\forall \alpha.\varphi\rightarrow\beta)\rightarrow \beta}$$

Результат:

\begin{center}\begin{tabular}{l}
$((\Lambda \beta.\lambda e^{\forall \alpha.\varphi\rightarrow\beta}.(e\ \theta)\ M)\ \psi)\ (\Lambda \alpha.\lambda x^\varphi.N) \rightarrow_\beta$\\
$(\lambda e^{\forall \alpha.\varphi\rightarrow\psi}.(e\ \theta)\ M)\ (\Lambda \alpha.\lambda x^\varphi.N) \rightarrow_\beta$\\
$(\Lambda \alpha.\lambda x^\varphi.N)\ \theta\ M \rightarrow_\beta$\\
$(\lambda x^{\varphi[\alpha:=\theta]}.N[\alpha:=\theta])\ M \rightarrow_\beta N[\alpha:=\theta][x := M]$
\end{tabular}\end{center}

\end{frame}

\begin{frame}[fragile]{Пример реализации на Хаскеле}
\footnotesize
\begin{verbatim}
{-# LANGUAGE RankNTypes #-}
data AbstractStack = AS (forall b . (forall a . 
     ( a, Integer -> a -> a, a -> (a, Integer) ) 
     -> b) -> b)

abstype :: AbstractStack -> Integer
abstype stack =
  case stack of
    AS r -> r x where
      x (empty, push, pop) =
        let (stk, v) = pop (push 12 $ push 5 empty) in
        let (stk2, v2) = pop stk in 
        v + v2

packedStack :: AbstractStack
packedStack = AS (\t -> t ( [], \i -> \l -> i:l, \(i:l) -> (l,i) ) )

main = do print (abstype packedStack)
\end{verbatim}
\end{frame}

\begin{frame}{Общие свойства системы $F$}
%\begin{thm}
В системе $F$ (в варианте по Чёрчу, так и в варианте по Карри) имеют место теорема Чёрча-Россера и сильная нормализация.\vspace{0.5cm}
%\end{thm}

Разрешимость задач типизации системы $F$:

\begin{center}\begin{tabular}{l|ll}
 & По Чёрчу & По Карри \\\hline
$\Gamma \vdash M:\sigma$ & да & нет\\
$\Gamma \vdash M:\ ?$ & да & нет\\
$\Gamma \vdash\ ? : \sigma$ & нет & нет\\
$? \vdash M:\sigma$ & нет & нет\\
$? \vdash M :\ ?$ & нет & нет
\end{tabular}\end{center}
\end{frame}

\begin{frame}{Ранг типа}
Напомним, что $\exists\alpha.\varphi := \forall \beta.(\forall\alpha.\varphi\rightarrow\beta)\rightarrow\beta$.
\begin{dfn}Функция <<ранг типа>> $rk \subseteq T \times \mathbb{N}_0$. $rk(\sigma) = [mrk(\sigma),+\infty)\cap\mathbb{N}_0$, где $mrk$:
%Зададим как функцию $rk: \tau \rightarrow \mathcal{P}(\mathbb{N}_0)$:
%$$rk(\tau) = \left\{\begin{array}{ll}
%\mathbb{N}_0 & \tau\text{ без кванторов}\\
%rk(\sigma) \setminus \{0\} & \tau = \forall x.\sigma\\
%\{ r + 1\ |\ r \in rk(\sigma_1)\} \cap rk(\sigma_2) & \tau = \sigma_1\rightarrow\sigma_2 
%\end{array}\right.$$

$$mrk(\tau)=\left\{\begin{array}{ll}
0, &\tau \text{ без кванторов}\\
\max(mrk(\sigma),1), &\tau = \forall x.\sigma\\
\max(mrk(\sigma_1)+1,mrk(\sigma_2)), & \tau = \sigma_1\rightarrow\sigma_2, \tau\text{ имеет кванторы}
\end{array}\right.
$$
\end{dfn}\vspace{-0.5cm}

\begin{lmm}
%\begin{enumerate}
%\item Если $rk(\tau,r)$, то $rk(\tau,s)$ для всех $s > r$;
Если $rk(\sigma,1)$, то для формулы $\sigma$ найдётся эквивалентная формула с поверхностными кванторами.
%\end{enumerate}
\end{lmm}

\begin{exm}
$0 \notin rk(\forall\alpha.\gamma\rightarrow\beta)$; $1 \notin rk((\forall\alpha.\gamma\rightarrow\beta)\rightarrow f)=\{2,3,\dots\}$\\
$1 \notin rk(\exists\alpha.\gamma) = rk(\forall\beta.(\forall\alpha.\gamma\rightarrow\beta)\rightarrow\beta) = \{2,3,\dots\}$\\
$1 \in rk(\forall\alpha.\delta\rightarrow\forall\beta.\delta\rightarrow\forall\gamma.\delta)$
%$(\forall x.\alpha)\rightarrow(\forall x.\beta)$ не имеет эквивалентной формулы с поверхностными кванторами.
\end{exm}

% $$rk(\tau)$$\end{dfn}
\end{frame}

\begin{frame}{Типовая система Хиндли-Милнера: язык}
\begin{dfn}Тип $(\tau)$ и типовая схема: $$\tau ::= \alpha\ |\ (\tau\rightarrow\tau)\quad\quad\sigma ::= \forall x.\sigma\ |\ \tau$$%\end{dfn}
%\begin{dfn}
Пред-лямбда-терм (типизация по Карри) $$H ::= x\ |\ (H\ H)\ |\ (\lambda x.H)\ |\ (\texttt{let }x=H\texttt{ in }H)$$%\end{dfn}
%\begin{dfn}
Редукция для let:
$$\texttt{let }x=E_1\texttt{ in }E_2\rightarrow_\beta E_2[x := E_1]$$
\end{dfn}

\begin{exm}
$$\texttt{let }Inc=\lambda n.\lambda f.\lambda x.n\ f\ (f\ x)\texttt{ in }Inc(Inc\ \overline{0})\twoheadrightarrow_\beta \overline{2}$$
\end{exm}
\end{frame}

\begin{frame}{Типовая система Хиндли-Милнера: специализация}
\begin{dfn}

Пусть $\sigma_1 = \forall \alpha_1. \forall \alpha_2. \dots \forall \alpha_n. \tau_1$.
Тогда $\sigma_2$ --- частный случай или специализация $\sigma_1$ (обознается как $\sigma_1 \sqsubseteq \sigma_2$), если

$$\sigma_2 =  \forall \beta_1. \forall \beta_2. \dots \forall \beta_m. \tau_1[\alpha_1 := S(\alpha_1),\dots,\alpha_n := S(\alpha_n)]
\text{ и } \beta_i \notin FV(\forall \alpha_1.\forall\alpha_2.\dots\forall\alpha_n.\tau_1)$$

\end{dfn}

\begin{exm}\

$\forall \alpha.\alpha \rightarrow \alpha
\sqsubseteq
\forall \beta_1.\forall \beta_2.(\beta_1 \rightarrow \beta_2) \rightarrow (\beta_1 \rightarrow \beta_2)$
\end{exm}
\end{frame}

\begin{frame}{Типовая система Хиндли-Милнера: правила вывода}\vspace{-0.7cm}
    $$\infer[x \not \in FV(\Gamma)]{\Gamma, x : \sigma \vdash x : \sigma}{}\qquad
    \infer{\Gamma \vdash E_0\ E_1 : \tau'}{\Gamma \vdash E_0 : \tau \rightarrow \tau' \qquad \Gamma \vdash E_1 : \tau}\qquad
    \infer{\Gamma \vdash \lambda x.E : \tau \rightarrow \tau'}{\Gamma, x : \tau \vdash E : \tau'}$$
    $$\infer{\Gamma \vdash let\ x = E_0\ in\ E_1 : \tau}{\Gamma \vdash E_0 : \sigma \qquad \Gamma, x : \sigma \vdash E_1 : \tau}\qquad
    \infer[\sigma' \sqsubseteq \sigma]{\Gamma \vdash E : \sigma}{\Gamma \vdash E : \sigma'}\qquad
    \infer[\alpha \not \in FV(\Gamma)]{\Gamma \vdash E : \forall \alpha.\sigma}{\Gamma \vdash E : \sigma}$$

%$$let\ x = a\ in\ b =_\beta (\lambda x.b)\ a$$

\vspace{-0.1cm}
\begin{exm}
\end{exm}\vspace{-0.8cm}
$$
\infer{\vdash\lambda x.x:\forall\alpha.\alpha\rightarrow\alpha}{\infer{\vdash\lambda x.x:\alpha\rightarrow\alpha}{\infer{x:\alpha\vdash x:\alpha}{}}}
$$
\vspace{-0.3cm}
$$\infer{\texttt{id}:\forall\alpha.\alpha\rightarrow\alpha\vdash\texttt{id}\ 0:\texttt{int}}
        {\infer[S(\alpha)=\texttt{int}]{\texttt{id}:\forall\alpha.\alpha\rightarrow\alpha\vdash\texttt{id}:\texttt{int}\rightarrow\texttt{int}}
               {\infer{\texttt{id}:\forall\alpha.\alpha\rightarrow\alpha\vdash\texttt{id}:\forall\alpha.\alpha\rightarrow\alpha}{}}\qquad
         \texttt{id}:\forall\alpha.\alpha\rightarrow\alpha\vdash 0:\texttt{int}}
$$


Отсюда: $\texttt{let id=}\lambda x.x\text{ in }\langle \texttt{id }0,\texttt{id <<a>>}\rangle:\texttt{int}\&\texttt{string}$

\end{frame}

\begin{comment}
\begin{frame}{Алгоритм реконструкции типа W}
На вход подаются $\Gamma,\ M$, на выходе наиболее общая пара: $\langle S, \tau \rangle = W(\Gamma,M)$
\begin{enumerate}
    \item $M = x ,\ x:\tau  \in \Gamma$ (иначе ошибка)
    \begin{itemize}
        \item $\tau'$ --- $\tau$ без кванторов, все свободные переменные переименованы в свежие.
        %Например: $\forall \alpha_1.\phi \Rightarrow \phi[\alpha_1 := \beta_1]$, где $\beta_1$ --- свежая переменная
    \end{itemize}
    возвращаем $\langle\emptyset, \tau'\rangle$; например, $W(\{x:\forall \alpha.\varphi, y:\beta\}, x) = \langle\emptyset,\varphi[\alpha:=\gamma]\rangle$
    \item $M = \lambda n.E$
    \begin{itemize}
        \item $\Gamma' = \{x : \sigma\ |\ x:\sigma \in \Gamma, x \ne n\} \cup \{n : \alpha\}$, $\alpha$ --- свежая типовая переменная
        \item $\langle S,\ \tau \rangle = W(\Gamma', E)$
    \end{itemize}
    возвращаем $\langle S, S(\alpha) \rightarrow \tau\rangle$
    \item $M = P\ Q$
    \begin{itemize}
        \item $\langle S_1, \tau_1 \rangle = W(\Gamma, P)$; $\langle S_2, \tau_2 \rangle = W(S_1(\Gamma), Q)$
        \item $S_3 = \mathcal{U}\big[S_2(\tau_1), \tau_2 \rightarrow \alpha\big]$, $\alpha$ --- свежая
    \end{itemize}
    возвращаем $\langle S_3 \circ S_2 \circ S_1, S_3(\alpha) \rangle$
    \item $M = (let\ n = P\ in\ Q)$
    \begin{itemize}
        \item $\langle S_1, \tau_1\rangle = W(\Gamma, P)$
        \item $\Gamma' = \{x:\sigma\ |\ x:\sigma\in\Gamma,x \ne n\} \cup \{ n : \forall \alpha_1 \dots \alpha_k. \tau_1 \}$, где $\alpha_1 \dots \alpha_k$ --- все свободные переменные $\tau_1$
        \item $\langle S_2, \tau_2\rangle = W(S_1(\Gamma'), Q)$
    \end{itemize}
    возвращаем $\langle S_2 \circ S_1, \tau_2 \rangle$
\end{enumerate}
\end{frame}

\begin{frame}{Рекурсия в HM: делаем HM тьюринг-полной}

\begin{enumerate}
\item Рекурсия для термов. $Y$-комбинатор. Добавим специальное правило вывода:

$$\infer{Y : \forall \alpha.(\alpha\rightarrow\alpha)\rightarrow\alpha}{}$$

\item Рекурсия для типов. Рассмотрим список

\vspace{-0.2cm}
$$\texttt{Nil} = In_L\ 0\qquad \texttt{Cons}\ e\ l = In_R \langle e,l \rangle\qquad \texttt{List} :\ ?$$

Заметим, что при попытке выписать уравнение для типа мы получим рекурсию:

\vspace{-0.5cm}
$$\tau = \texttt{Int} \vee \langle \texttt{Int}, \tau \rangle$$

Рекурсивный тип надо добавить явно: $$\tau = \mu \alpha.\texttt{Int}\vee\langle\texttt{Int},\alpha\rangle$$
Мю-оператор --- это Y-комбинатор для типов. Как его добавить в типовую систему?

\end{enumerate}
\end{frame}

\begin{frame}[fragile]{Эквирекурсивные и изорекурсивные типы: $\mu\alpha.\sigma(\alpha)$}
\begin{itemize}
\item Эквирекурсивные типы. Считаем, что $\alpha = \sigma(\alpha)$.
Например, в Java:
\begin{verbatim}
public abstract class Enum<E extends Enum<E>>
        implements Constable, Comparable<E>, Serializable 
{ ... }
\end{verbatim}

Уравнение (частный случай): $E = Enum(E)$, или $E = \mu \varepsilon . Enum(\varepsilon)$.

\item Изорекурсивные типы. $\alpha \ne \sigma(\alpha)$, но есть изоморфизм:
 $$\texttt{roll}: \sigma(\alpha)\rightarrow\alpha\quad\quad\texttt{unroll}: \alpha\rightarrow\sigma(\alpha)$$

Например, для \verb!struct List { List* next; int value; }!:

\begin{center}\begin{tabular}{lll}
Комп. & В C++ & Пример \\\hline
\verb!roll! & взятие ссылки & \small\verb!List a; a.next = NULL; return len(!\color{blue}\verb!&a!\color{black}\verb!)!\\
\verb!unroll! & разыменование & \small\verb!len (List* a) { return (!\color{blue}\verb!*a!\color{black}\verb!).next ? ... : 0 }!
\end{tabular}\end{center}

\end{itemize}
\end{frame}

\begin{frame}{Разрешимость задачи реконструкции типа в разных вариантах $F$}
\begin{tabular}{lll}
Ранг типов & Собственное название & Разрешимость\\\hline
0 & $\lambda_\rightarrow$ & разрешимо (лекция 2)\\
1 & $HM$ & разрешимо (алгоритм $W$)\\
2 & & разрешимо\\
$\ge3$ & & неразрешимо
\end{tabular}
\end{frame}
\end{comment}

\end{document}
\end{document}
